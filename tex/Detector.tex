\section{ATLAS Detector}
\label{sec:atlasdet}
\paragraph{} The ATLAS detector is one of two general detectors at LHC. It is normally forward-backward symmetric and covers nearly all the solid angle around proton-proton interaction point. From inner it consists of an inner tracker detector(ID), a superconducting solenoid, electromagnetic and hadronic calorimeter, and a muon spectrometer. 3 main component of Inner Detector, pixel detector, semiconductor tracker and transition radiation tracker provide precise measurement of transverse momentum, direction, charge for the charged particle, as well as a preliminary identification to b-jets in barrel region $(|\eta|<2.5)$. The whole Inner Detector is immersed in a 2T axial magnetic field provided by the solenoid. The outer part, electromagnetic calorimeter is a high granularity lead/liquid-argon(LAr) sampling calorimeter, measures the electromagnetic shower in barrel $(|\eta|<1.475)$ and endcap $(1.375<|\eta|<3.2)$ regions. Following hadronic calorimeter can reconstruct the hadronic shower with steel and scintillator tiles $(|\eta|<1.7)$, copper/LAr $(1.5<|\eta|<2.7)$ or copper-tungsten/LAr $(3.1<|\eta|<4.9)$ respectively. The outermost layer muon spectrometer comprises 3 large superconducting eight-coil toroids, separate trigger chambers $(|\eta|<2.4)$ and precision tracking chambers $(|\eta|<2.7)$.

\paragraph{} The ATLAS data-taking system uses a two-level system: a hardware-based level trigger(L1) to reduce the event rate to at most 100 kHz, and a software-based level trigger(L2) to reduce the event rate to approximately 1 kHz. 

