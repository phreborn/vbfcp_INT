\section{Data and Monte Carlo Samples}
\label{sec:dataMC}

\paragraph{} This analysis uses LHC proton-proton collision data collected by ATLAS detector from 2015 to 2018, at central\todo{center}-of-mass energy $\sqrt(s)=13~\TeV$. After requiring good data quality the data set amounts to integrated luminosity 139.8 \ifb \todo{add GRL tags}. The mean number of interactions per bunch crossing, $\mu$, was 34 on average, varying from $24$ in 2015-2016 to $37$ in 2018 data. \todo{trigger should be a dedicated paragraph in next section} Events were selected by a di-photon trigger with pT thresholds of 35GeV and 25 GeV for leading and sub-leading photon. A tight photon identification and isolation requirement are applied. On average the trigger efficiency can reach to 98\% after the event selection \todo{cite!! also for lumi unc.}.

\begin{table}[htbp]
\centering
\begin{tabular}{l|c}
\hline
Period & $\int \mathcal{L} dt [fb^{-1}]$ \\ \hline
2015   & 3.2                   \\
2016   & 32.9                  \\
2017   & 43.8                  \\
2018   & 59.9                  \\ \hline
total  & 139.8                 \\
\hline
\end{tabular}
\caption{Integrated luminosity in ATLAS RunII period. The combined luminosity uncertainty is 2.0\% }
\label{tab:dataLumi}
\end{table}

\paragraph{} The main Standard Model processes considered in this work are gluon-fusion (ggF) Higgs production, Vector Boson Fusion (VBF) Higgs production, and continuum diphoton production. For the ggF and VBF process, Monte-Carlo (MC) generator provides a precise simulation, the detailed generator, parton distribution functions (PDFs) and perturbative order in QCD is summarized in Table ~\ref{tab:MC} \todo{VBF is generated at NLO but normalized to approximate-NNLO, ggF is generated at NNLO and normalized to N3LO} \todo{in the table add the stats in these samples for each mc campaign, add an appendix detailing the full DAOD names}

\begin{center}
\centering
\begin{table}[htbp]
\begin{tabular}{l|l|l|l|l|l}
\hline
Process        & Generator & Showering & PDF sets  & QCD order         & $\sigma [pb] \sqrt{s}=13TeV$ \\
\hline
VBF            & Powheg    & PYTHIA8   & PDF4LHC15 & NNLO(QCD)+NLO(EW) & 3.75                         \\
ggF            & Powheg    & PYTHIA8   & PDF4LHC15 & NNLO(QCD)+NLO(EW) & 28.3                         \\
$\gamma\gamma$ & Sherpa    & Sherpa    & CT10      &                   &                             	\\
\hline
\end{tabular}
\caption{Monte Carlo samples used in this analysis.}
\label{tab:MC}
\end{table}
\end{center}


\todo{this paragraph should be moved to an analysis strategy section in the introduction or to a dedicated chapter before event selection, detailing that to avoid any model dependence on the 7-phase space variables a fiducial-like measurement is performed using data sideband for estimating the background.}
\paragraph{}The complicated final state in di-photon + multi-jets process makes it hard to simulate in MC. The Optimal Observable combines the information from 7-dimension phase space, any mismodelling would lead to an biased estimation in OO. So sideband data becomes the best candidate for the background, for both the modelling and event yield. While restricted by the statistics, di-photon continuum background modelling could not be performed by sideband data, so a set of di-photon+jets process MC are generated by Sherpa, with proper hadronization and reconstruction. 

\paragraph{}To simulate the CP violation phenomenon in HVV vertex, a matrix element based reweighting method is performed in the SM VBF sample. The weight is defined by the square of matrix element value of VBF process associated with a specific amount of CP mixing $\tilde(d)$ to the one obtained from SM. The corresponding 2 matrix elements are calculated by HAWK with initial and final state information in each event, assuming a $2\to2+H$ or $2\to3+H$ process. For convenience the weight is parameterized as a function of \todo{define tilde d as $\tilde{d}$} $\tilde(d)$, so the CPV process in any mixing level could be simulated immediately. \todo{explain what is w1 and w2}.

\begin{center}
\begin{math}
w = 1+\tilde{d}w_1 + \tilde{d}^2w_2
\end{math}
\end{center}

\paragraph{} For all \todo{incorrect, Sherpa is fast sim} MC generated samples, a full simulation of ATLAS detector response with Geant4 program is performed, except the CPV VBF Higgs process is reweighted from reconstructed SM VBF process. Additional proton-proton interactions (pile-up) are produced using Pythia8 with A2 parameter set and MSTW2008LO PDF set. They are included in the simulation for all generated events such that the distribution of the mean number of interactions per bunch crossing reproduces that observed in the data.

