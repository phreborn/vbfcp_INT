\section{Signal and background modelling}
\label{sec:model}


\subsection{Signal modelling}
\label{subsec:sigmodel}

\subsubsection{Signal yield for SM and BSM}
\label{sssec:sigyield}
\paragraph{} As discussed in Sec [ref:theory], the VBF event yields could be influenced by the additional CP-violation EFT terms. The branching ration of Higgs to diphoton $Br(H\to\gamma\gamma)$ varies from 0.227\% in SM to above 27\% in $\tilde{d}=\pm 0.01$ so that it can have a extremely strict restriction to this EFT model. Considering that the VBF process cross section can be also influenced by some other CP-conserving new physics, in the main part of this analysis ignored all restrictions from VBF event yields and only considered the OO distribution. A very preliminary result performed by both signal yield and OO shape is discussed in Appendix [ref:AppendixBr]. 

\paragraph{} The event yields for each SM process in every OO bins are displayed in Table ~\ref{tab:NevtSM}. VBF signal yields for different $\tilde{d}$ in each OO bins are in Table ~\ref{tab:NevtBSM}, total event number in all bins is scaled to SM case. These are extracted from Monte Carlo samples. 

\begin{table}[h!]
\begin{center}
\begin{tabular}{l|cccccc}
\hline
        & bin1  & bin2  & bin3  & bin4  & bin5  & bin6  \\ \hline
SM VBF  & 24.18 & 19.83 & 22.63 & 22.59 & 19.94 & 24.14 \\
ggH     & 10.70 & 12.33 & 26.16 & 26.25 & 12.25 & 10.58 \\
$\gamma\gamma$ & 1  & 1 & 1     & 1     & 1     & 1     \\ 
SB data & 1436  & 1662  & 3111  & 3048  & 1737  & 1474  \\
\hline
\end{tabular}
\caption{Event yields in 6 OO bins for each SM process and sid-band data. \textcolor{red}{h024 result. Need update to h025.}}
\label{tab:NevtSM}
\end{center}
\end{table}

\begin{table}[ht]
\begin{center}
\begin{tabular}{l|cccccc}
\hline
      & bin1  & bin2  & bin3  & bin4  & bin5  & bin6  \\ \hline
-0.1  & 47.18 & 27.37 & 26.20 & 21.50 & 16.52 & 14.77 \\
-0.09 & 44.27 & 26.43 & 25.73 & 21.50 & 16.68 & 15.11 \\
-0.08 & 41.49 & 25.53 & 25.29 & 21.52 & 16.87 & 15.57 \\
-0.07 & 38.85 & 24.67 & 24.87 & 21.57 & 17.11 & 16.18 \\
-0.06 & 36.35 & 23.86 & 24.48 & 21.64 & 17.39 & 16.91 \\
-0.05 & 33.98 & 23.08 & 24.11 & 21.73 & 17.71 & 17.78 \\
-0.04 & 31.74 & 22.35 & 23.76 & 21.86 & 18.07 & 18.78 \\
-0.03 & 29.65 & 21.66 & 23.44 & 22.00 & 18.47 & 19.92 \\
-0.02 & 27.69 & 21.01 & 23.15 & 22.17 & 18.92 & 21.19 \\
-0.01 & 25.87 & 20.40 & 22.88 & 22.37 & 19.41 & 22.60 \\
0     & 24.18 & 19.83 & 22.63 & 22.59 & 19.94 & 24.14 \\
0.01  & 22.63 & 19.30 & 22.41 & 22.84 & 20.51 & 25.82 \\
0.02  & 21.22 & 18.82 & 22.22 & 23.11 & 21.12 & 27.63 \\
0.03  & 19.94 & 18.37 & 22.05 & 23.41 & 21.78 & 29.57 \\
0.04  & 18.80 & 17.97 & 21.90 & 23.73 & 22.48 & 31.65 \\
0.05  & 17.79 & 17.60 & 21.78 & 24.08 & 23.22 & 33.86 \\
0.06  & 16.93 & 17.28 & 21.68 & 24.45 & 24.00 & 36.20 \\
0.07  & 16.19 & 17.00 & 21.61 & 24.85 & 24.82 & 38.68 \\
0.08  & 15.60 & 16.76 & 21.56 & 25.27 & 25.69 & 41.30 \\
0.09  & 15.14 & 16.56 & 21.54 & 25.72 & 26.59 & 44.05 \\
0.1   & 14.82 & 16.41 & 21.54 & 26.19 & 27.54 & 46.93 \\
\hline
\end{tabular}
\caption{BSM VBF event yields in 6 OO bins, $\tilde{d}$ varies from -0.1 to 0.1. \textcolor{red}{h024 result. Need update to h025.} }
\label{tab:NevtBSM}
\end{center}
\end{table}



\subsubsection{Signal shape}
\label{sssec:sigshape}
\paragraph{} In experiment the Higgs boson signal is described with a double-side Crystal Ball function, with the bulk modeled by a Gaussian distribution and lower tails by two independent power-low functions[ref:DSCB]. The parameters in the model are determined through fitting simulated Higgs process in each reconstructed VBF-enriched categories and OO bins. The uncertainties from signal model are considered as nuisance parameter in final systematic uncertainty estimation. 


\subsection{Background modelling}
\label{subsec:bkgmodel}
\paragraph{} The main background in this analysis are non-resonant continuum background from inclusive $\gamma\gamma$ process, and corresponding di-photon final state with one or two photons from fake jets, calling $\gamma j$ and $jj$ events. These have smoothly falling distribution in the interested $m_{\gamma\gamma}$ region and described by an empirically chosen functional form. The fraction of 3 compositions is estimated with 2x2D sideband method, while the shapes are studied as discussed in chapter [ref:bkgshape]. Then the spurious signal test is performed for determining the background model individually in each OO bins. Appendix ~\ref{appendix:2x2DSB} summaries the results of background decomposition. 

\subsubsection{Background templates derivation}
\label{ssssec:bkgtemplate}

\paragraph{} Among the three components, $\gamma\gamma$ process could be simulated with Monte Carlo, fake jet performance in $\gamma j$ and $jj$ can only be studied from data in control region due to computational limitation. Here $\gamma-j$ and $jj$ are considered together to simplify the work [\ref:HGamfiducial140]. The control region is built by inverting the tight photon identification and isolation criteria on one of the photon candidates in the final state, while all other selections are the nominal ones. This ensures enough statistics in each OO bins. $\gamma\gamma$ contamination in this control region is estimated with MC simulation and is subtracted from data, which is only a few percent. 

\paragraph{} A reweighting is performed to have further lower statistical fluctuation in this reducible background component. A smoothing function from fitting the ratio of the above control region data distribution and MC $\gamma\gamma$ distribution for events after nominal selections is used as the weight function, and this weight is applied to a high-statistics $\gamma\gamma$ MC sample. 2nd order polynomial function is selected as the fit function to the ratio. \\

\paragraph{} The total background template consists of simulated $\gamma\gamma$ and this smoothed $\gamma-j$ process. Relative fraction is calculated by 2x2D sideband method in each OO bins. The overall yield is normalized to match the side-band data in signal region. There are 3 sources of uncertainty in this template: statistics uncertainty of $\gamma\gamma$ MC, uncertainty of fraction from 2x2D, and the uncertainty of smoothing function. The first two terms are easily to obtain, the last one comes from the $m_\gamma\gamma$ distribution difference with 1st order and 2nd order polynomial as smoothing function. All these 3 terms are combined together in GPR smooth to have final background template. 

\begin{figure}[htbp]
  \centering
  \subfloat[$bin1: OO<-2$]{
  \includegraphics[width=.32\textwidth]{figure/bkg_templates/1_control_region/invID_invIso_b1.png}} 
  \subfloat[$bin2: -2<OO<-1$]{
  \includegraphics[width=.32\textwidth]{figure/bkg_templates/1_control_region/invID_invIso_b2.png}}
  \subfloat[$bin3: -1<OO<0 $]{
  \includegraphics[width=.32\textwidth]{figure/bkg_templates/1_control_region/invID_invIso_b3.png}} \\
  \subfloat[bin4: $0<OO<1$ ]{
  \includegraphics[width=.32\textwidth]{figure/bkg_templates/1_control_region/invID_invIso_b4.png}}
  \subfloat[bin5: 1<OO<2]{
  \includegraphics[width=.32\textwidth]{figure/bkg_templates/1_control_region/invID_invIso_b5.png}}
  \subfloat[bin6: 2<OO]{
  \includegraphics[width=.32\textwidth]{figure/bkg_templates/1_control_region/invID_invIso_b6.png}} \\
  \caption{Contral region sample used to estimate $\gamma j+jj$ background shape. $\gamma\gamma$ contamination has been subtracted with MC. }
  \label{fig:bkg_CR}
\end{figure}

\begin{figure}[htbp]
  \centering
  \subfloat[bin1: OO<-2]{
  \includegraphics[width=.32\textwidth]{figure/bkg_templates/2_reweight_function/invID_invIso_smooth_ratio_b1.png}} 
  \subfloat[bin2: -2<OO<-1]{
  \includegraphics[width=.32\textwidth]{figure/bkg_templates/2_reweight_function/invID_invIso_smooth_ratio_b2.png}} 
  \subfloat[bin3: -1<OO<0 ]{
  \includegraphics[width=.32\textwidth]{figure/bkg_templates/2_reweight_function/invID_invIso_smooth_ratio_b3.png}} \\
  \subfloat[bin4: 0<OO<1]{
  \includegraphics[width=.32\textwidth]{figure/bkg_templates/2_reweight_function/invID_invIso_smooth_ratio_b4.png}} 
  \subfloat[bin5: 1<OO<2]{
  \includegraphics[width=.32\textwidth]{figure/bkg_templates/2_reweight_function/invID_invIso_smooth_ratio_b5.png}} 
  \subfloat[bin6: 2<OO]{
  \includegraphics[width=.32\textwidth]{figure/bkg_templates/2_reweight_function/invID_invIso_smooth_ratio_b6.png}} \\
  \caption{Contral region data smoothed by a reweighting method.}
  \label{fig:bkg_reweight}
\end{figure}

\begin{figure}[htbp]
  \centering
  \subfloat[bin1: OO<-2]{
  \includegraphics[width=.32\textwidth]{figure/bkg_templates/3_raw_template/invID_invIso_uncertainty_b1.png}} 
  \subfloat[bin2: -2<OO<-1]{
  \includegraphics[width=.32\textwidth]{figure/bkg_templates/3_raw_template/invID_invIso_uncertainty_b2.png}} 
  \subfloat[bin3: -1<OO<0 ]{
  \includegraphics[width=.32\textwidth]{figure/bkg_templates/3_raw_template/invID_invIso_uncertainty_b3.png}} \\
  \subfloat[bin4: 0<OO<1]{
  \includegraphics[width=.32\textwidth]{figure/bkg_templates/3_raw_template/invID_invIso_uncertainty_b4.png}} 
  \subfloat[bin5: 1<OO<2]{
  \includegraphics[width=.32\textwidth]{figure/bkg_templates/3_raw_template/invID_invIso_uncertainty_b5.png}} 
  \subfloat[bin6: 2<OO]{
  \includegraphics[width=.32\textwidth]{figure/bkg_templates/3_raw_template/invID_invIso_uncertainty_b6.png}} \\
  \caption{Total background templates built with $\gamma\gamma, \gamma j+jj$ components.}
  \label{fig:bkg_total}
\end{figure}


\subsubsection{GPR smooth}
\label{sssub:GPRsmooth}

\paragraph{} In order to further remove the statistical fluctuations, the background template is smoothed with a Gaussian Process Regression (GPR) techniques. Two hyper parameters for GPR smoothing, length scale $\lambda$ and length scale slope $b_\lambda$, are optimized with a 2D space scan, which requires that smoothing out at least 33\% of narrow bump(O(bin width)) and smoothing out less than 25\% wide bump(O(signal width)) in $m_{\gamma\gamma}$ distribution. Figure ~\ref{fig:GPRdis} shows the $m_{\gamma\gamma}$ distribution before and after GPR smoothing method, comparing with the sideband data. Figure ~\ref{fig:GPRopt} shows the optimized hyper parameter spaces. 

\begin{figure}[htbp]
  \centering
  \subfloat[bin1: OO<-2]{
  \includegraphics[width=.32\textwidth]{figure/bkg_templates/4_smoothed_template/GPR_Smoothed_Plot_template_uncer_b1.png}} 
  \subfloat[bin2: -2<OO<-1]{
  \includegraphics[width=.32\textwidth]{figure/bkg_templates/4_smoothed_template/GPR_Smoothed_Plot_template_uncer_b2.png}} 
  \subfloat[bin3: -1<OO<0 ]{
  \includegraphics[width=.32\textwidth]{figure/bkg_templates/4_smoothed_template/GPR_Smoothed_Plot_template_uncer_b3.png}} \\
  \subfloat[bin4: 0<OO<1]{
  \includegraphics[width=.32\textwidth]{figure/bkg_templates/4_smoothed_template/GPR_Smoothed_Plot_template_uncer_b4.png}} 
  \subfloat[bin5: 1<OO<2]{
  \includegraphics[width=.32\textwidth]{figure/bkg_templates/4_smoothed_template/GPR_Smoothed_Plot_template_uncer_b5.png}} 
  \subfloat[bin6: 2<OO]{
  \includegraphics[width=.32\textwidth]{figure/bkg_templates/4_smoothed_template/GPR_Smoothed_Plot_template_uncer_b6.png}} \\
  \caption{Background template distribution after GPR smoothing.}
  \label{fig:GPRdis}
\end{figure}


\begin{figure}[htbp]
  \centering
  \subfloat[bin1: OO<-2]{
  \includegraphics[width=.32\textwidth]{figure/bkg_templates/GPR/GoodHyperPars_bkg_template_b1.png}} 
  \subfloat[bin2: -2<OO<-1]{
  \includegraphics[width=.32\textwidth]{figure/bkg_templates/GPR/GoodHyperPars_bkg_template_b2.png}} 
  \subfloat[bin3: -1<OO<0 ]{
  \includegraphics[width=.32\textwidth]{figure/bkg_templates/GPR/GoodHyperPars_bkg_template_b3.png}} \\
  \subfloat[bin4: 0<OO<1]{
  \includegraphics[width=.32\textwidth]{figure/bkg_templates/GPR/GoodHyperPars_bkg_template_b4.png}} 
  \subfloat[bin5: 1<OO<2]{
  \includegraphics[width=.32\textwidth]{figure/bkg_templates/GPR/GoodHyperPars_bkg_template_b5.png}} 
  \subfloat[bin6: 2<OO]{
  \includegraphics[width=.32\textwidth]{figure/bkg_templates/GPR/GoodHyperPars_bkg_template_b6.eps}} \\
  \caption{Optimized GPR hyper parameter spaces}
  \label{fig:GPRopt}
\end{figure}


\subsubsection{Spurious signal test}
\label{sssub:sstest}

\paragraph{} The final choice of background model depends on a spurious signal test. This test is done by fitting a background only shape with a signal+background function, the extracted signal yield (S) and its statistics uncertainty ($\Delta_S$) are used to measure the bias introduced by the choice of background model. The function could be accepted if it satisfy one of the following criteria: 

\begin{itemize}
\item{} $S\pm \Delta_S $ is less than 20\% of the background uncertainty
\item{} $S\pm \Delta_S $ is less than 10\% of expected signal event number. 
\end{itemize}

Tested functions include 1st, 2nd, 3rd exponential, 3,4,5 order Bernstein polynomial and first order power low function. In case not only one functions pass the criteria, the one with the fewest degree of freedom is chosen. Table ~\ref{tab:SSbin1} \~ ~\ref{tab:SSbin6} summarized the spurious signal test results and chosen background functions in 6 OO bins. 

\clearpage
\begin{landscape}

\begin{table}[]
\footnotesize
\begin{tabular}{l|ccccccccccc}
Name        & max(S/deltaS) {[}\%{]} & max(S/Sref) {[}\%{]} & max(S) & max(S) Err & nPars & chi2/ndof & Prob(chi2) {[}\%{]} & Stat Err & Stat Err {[}\%{]} & Relative Tot Err {[}\%{]} & passT0 \\ \hline
Exponential & 9.67                   & 4.84                 & 1.69   & 17.5       & 1     & 0.0327    & 100                 & 16.6     & 47.6              & 47.9                      & 1      \\
ExpPoly2    & 2.71                   & 2.56                 & 0.492  & 18.2       & 2     & 0.0134    & 100                 & 17.9     & 51.3              & 51.4                      & 1      \\
ExpPoly3    & -3.64                  & -2.15                & -0.642 & 17.6       & 3     & 0.00984   & 100                 & 18.8     & 53.9              & 53.9                      & 1      \\
Bern3       & -3.72                  & -2.49                & -0.663 & 17.8       & 3     & 0.0117    & 100                 & 19       & 54.6              & 54.6                      & 1      \\
Bern4       & -1.47                  & -1.08                & -0.276 & 18.8       & 4     & 0.0058    & 100                 & 19       & 54.6              & 54.6                      & 1      \\
Bern5       & 1.6                    & 1.35                 & 0.341  & 21.3       & 5     & 0.00178   & 100                 & 21.3     & 61.2              & 61.2                      & 1      \\
Pow         & 29.5                   & 14.1                 & 4.96   & 16.9       & 1     & 0.406     & 100                 & 16.6     & 47.6              & 49.6                      & 1      \\
\end{tabular}
\caption{Spurious signal test results for OO bin1. Exponential function is chosen.}
\label{tab:SSbin1}
\end{table}

\begin{table}[]
\footnotesize
\begin{tabular}{l|ccccccccccc}
Name        & max(S/deltaS) {[}\%{]} & max(S/Sref) {[}\%{]} &  max(S) & max(S) Err & nPars & chi2/ndof & Prob(chi2) {[}\%{]} & Stat Err & Stat Err {[}\%{]} & Relative Tot Err {[}\%{]} & passT0 \\ \hline
Exponential & 12.2                   & 6.36                 &  2.04   & 16.7       & 1     & 0.092     & 100                 & 17.6     & 54.7              & 55.1                      & 1      \\
ExpPoly2    & 6.03                   & 3.44                 &  1.11   & 18.4       & 2     & 0.0827    & 100                 & 19.1     & 59.4              & 59.5                      & 1      \\
Bern3       & -15.9                  & -10.4                &  -3.35  & 21.1       & 3     & 0.0384    & 100                 & 20.2     & 62.8              & 63.7                      & 1      \\
ExpPoly3    & -17.8                  & -12.1                &  -3.88  & 21.8       & 3     & 0.0457    & 100                 & 19.9     & 61.8              & 63                        & 1      \\
Bern4       & -15.5                  & -10.5                &  -3.35  & 21.6       & 4     & 0.0367    & 100                 & 20.2     & 62.8              & 63.6                      & 1      \\
Bern5       & -12.3                  & -8.71                &  -2.8   & 22.8       & 5     & 0.0227    & 100                 & 22.5     & 69.9              & 70.5                      & 1      \\
Pow         & 47.9                   & 24.9                 &  8.04   & 16.8       & 1     & 0.629     & 98.4                & 17.6     & 54.8              & 60.2                      & 1      \\
\end{tabular}
\caption{Spurious signal test results for OO bin2. Exponential function is chosen.}
\label{tab:SSbin2}
\end{table}

\begin{table}[]
\footnotesize
\begin{tabular}{l|ccccccccccc}
Name        & max(S/deltaS) {[}\%{]} & max(S/Sref) {[}\%{]} & max(S) & max(S) Err & nPars & chi2/ndof & Prob(chi2) {[}\%{]} & Stat Err & Stat Err {[}\%{]} & Relative Tot Err {[}\%{]} & passT0 \\ \hline
ExpPoly2    & -5.02                  & -3.26                & -1.31  & 26.2       & 2     & 0.00709   & 100                 & 25.3     & 51.8              & 51.9                      & 1      \\
ExpPoly3    & -3.08                  & -2.02                & -0.808 & 26.9       & 3     & 0.0047    & 100                 & 25.6     & 52.5              & 52.5                      & 1      \\
Bern3       & -6.48                  & -3.78                & -1.83  & 28.2       & 3     & 0.00628   & 100                 & 26.7     & 54.7              & 54.8                      & 1      \\
Bern4       & -2.85                  & -2.39                & -0.85  & 29.8       & 4     & 0.00236   & 100                 & 26.7     & 54.7              & 54.7                      & 1      \\
Bern5       & -3.66                  & -2.28                & -1.11  & 30.3       & 5     & 0.00179   & 100                 & 29.6     & 60.8              & 60.8                      & 1      \\
Pow         & 22.7                   & 10.4                 & 5.09   & 22.8       & 1     & 0.172     & 100                 & 23.2     & 47.5              & 48.6                      & 1      \\
Exponential & -35.9                  & -17.1                & -8.29  & 23.3       & 1     & 0.232     & 100                 & 23.1     & 47.4              & 50.3                      & 1     \\
\end{tabular}
\caption{Spurious signal test results for OO bin3. ExpPoly2 function is chosen.}
\label{tab:SSbin3}
\end{table}
\end{landscape}
\clearpage

\begin{landscape}
\begin{table}[]
\footnotesize
\begin{tabular}{l|ccccccccccc}
Name        & max(S/deltaS) {[}\%{]} & max(S/Sref) {[}\%{]} & max(S) & max(S) Err & nPars & chi2/ndof & Prob(chi2) {[}\%{]} & Stat Err & Stat Err {[}\%{]} & Relative Tot Err {[}\%{]} & passT0 \\ \hline
Exponential & -10.5                  & -4.71                & -2.3   & 21.9       & 1     & 0.0313    & 100                 & 23.4     & 48                & 48.2                      & 1      \\
ExpPoly2    & 12.4                   & 6.59                 & 3.22   & 25.9       & 2     & 0.0065    & 100                 & 25.6     & 52.4              & 52.8                      & 1      \\
Bern3       & 10.2                   & 5.59                 & 2.75   & 27.2       & 3     & 0.0087    & 100                 & 27       & 55.2              & 55.5                      & 1      \\
ExpPoly3    & 11                     & 6.12                 & 2.99   & 27.2       & 3     & 0.00599   & 100                 & 26.4     & 54.1              & 54.4                      & 1      \\
Bern4       & 11.2                   & 5.98                 & 3.05   & 27.6       & 4     & 0.00687   & 100                 & 27       & 55.2              & 55.6                      & 1      \\
Bern5       & 6.88                   & 4.27                 & 2.07   & 30.1       & 5     & 0.00351   & 100                 & 29.8     & 61                & 61.2                      & 1      \\
Pow         & 57.6                   & 27.9                 & 13.7   & 23.9       & 1     & 0.355     & 100                 & 23.5     & 48.1              & 55.6                      & 1      \\
\end{tabular}
\caption{Spurious signal test results for OO bin4. Exponential function is chosen.}
\label{tab:SSbin4}
\end{table}

\begin{table}[]
\footnotesize
\begin{tabular}{l|ccccccccccc}
Name        & max(S/deltaS) {[}\%{]} & max(S/Sref) {[}\%{]} &  max(S) & max(S) Err & nPars & chi2/ndof & Prob(chi2) {[}\%{]} & Stat Err & Stat Err {[}\%{]} & Relative Tot Err {[}\%{]} & passT0 \\ \hline
Exponential & 8                      & 4.81                 &  1.55   & 19.3       & 1     & 0.105     & 100                 & 18       & 56                & 56.2                      & 1      \\
ExpPoly2    & 7.04                   & 4.45                 &  1.43   & 20.3       & 2     & 0.107     & 100                 & 19.6     & 60.8              & 60.9                      & 1      \\
Bern3       & -11.8                  & -7.5                 &  -2.43  & 20.7       & 3     & 0.035     & 100                 & 20.7     & 64.3              & 64.7                      & 1      \\
ExpPoly3    & -12.5                  & -8.26                &  -2.66  & 21.4       & 3     & 0.0264    & 100                 & 20.4     & 63.3              & 63.8                      & 1      \\
Bern4       & -12.1                  & -8.12                &  -2.58  & 21.3       & 4     & 0.0316    & 100                 & 20.7     & 64.3              & 64.8                      & 1      \\
Bern5       & -6.35                  & -4.27                &  -1.48  & 23.4       & 5     & 0.00661   & 100                 & 23.1     & 71.6              & 71.8                      & 1      \\
Pow         & 39.2                   & 23.3                 &  7.56   & 19.3       & 1     & 0.586     & 99.3                & 18       & 56                & 60.7                      & 1      \\
\end{tabular}
\caption{Spurious signal test results for OO bin5. Exponential function is chosen.}
\label{tab:SSbin5}
\end{table}

\begin{table}[]
\footnotesize
\begin{tabular}{l|ccccccccccc}
Name        & max(S/deltaS) {[}\%{]} & max(S/Sref) {[}\%{]} & max(S) & max(S) Err & nPars & chi2/ndof & Prob(chi2) {[}\%{]} & Stat Err & Stat Err {[}\%{]} & Relative Tot Err {[}\%{]} & passT0 \\ \hline
Exponential & 6.56                   & 3.43                 & 1.05   & 16         & 1     & 0.0258    & 100                 & 16.6     & 47.9              & 48                        & 1      \\
ExpPoly2    & -6.2                   & -3.25                & -1.13  & 18.2       & 2     & 0.00975   & 100                 & 17.9     & 51.6              & 51.7                      & 1      \\
ExpPoly3    & -2.38                  & -1.77                & -0.487 & 20.4       & 3     & 0.00556   & 100                 & 18.8     & 54.1              & 54.1                      & 1      \\
Bern3       & -3.17                  & -2.26                & -0.629 & 19.8       & 3     & 0.00735   & 100                 & 19.1     & 54.9              & 55                        & 1      \\
Bern4       & -1.58                  & -1.42                & -0.325 & 20.6       & 4     & 0.00178   & 100                 & 19.1     & 55                & 55                        & 1      \\
Bern5       & 1.1                    & 0.967                & 0.222  & 20.3       & 5     & 0.00138   & 100                 & 21.4     & 61.6              & 61.6                      & 1      \\
Pow         & 31                     & 14.2                 & 4.97   & 16.1       & 1     & 0.341     & 100                 & 16.6     & 47.9              & 50                        & 1      \\
\end{tabular}
\caption{Spurious signal test results for OO bin6. Exponential function is chosen.}
\label{tab:SSbin6}
\end{table}

\end{landscape}




\subsection{F-test}
\label{subsec:Ftest}

