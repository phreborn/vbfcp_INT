\section{Introduction}
\label{sec:intro}


\paragraph{}The discovery of Higgs boson by ATLAS and CMS experiment in 2012 opens a new era for particle physics, and provides a new opportunity to search for new physics beyond Standard Model, such as the new source of CP-violation required by the observed baryon asymmetry in the universe. In the Standard Model the expected Higgs boson is a scalar boson with $J^{pc} = 0{++}$, till now all the experimental results in ATLAS and CMS are consistent with this prediction within the uncertainty.  However, limited by the statistics, the direct Higgs CP measurements are performed only in very recent period  \cite{HIGG-VBFHtautau}\cite{HIGG-ttHyy}\cite{CMS-HIG-17-034}\cite{CMS-HIG-19-013}. All of these observations show no derivation from SM, but the restriction is not so satisfying. 


\paragraph{} The focus of this work is the CP property of Higgs boson in its interaction with gauge vector bosons. The test is performed through the largest HVV production mode in LHC, Vector Boson Fusion, in diphoton decay channel, with 140 $fb^{-1}$ p-p collision data collected by ATLAS detector at $\sqrt(s)=13TeV$ during 2015 to 2018. In order to have a more sensitive result and be independent with CP-even observables, the Optimal Observable is introduced as in previous di-tau channel analysis. 



\paragraph{}The Effective Field Theory and Optimal Observable are introduced in Sec.~\ref{sec:theory}. Sec.~\ref{sec:atlasdet} and Sec.~\ref{sec:dataMC} show a brief description about ATLAS detector and the simulated data we used in this analysis. In Sec.~\ref{sec:hgam_selection} Sec.~\ref{sec:signal_model} and Sec.~\ref{sec:background_model}, we presented the main analyses procedure, including the event selection, categorization and event modelling. Following in Sec.~\ref{sec:syst} is the experimental uncertainty considered, and statistic model constructed for parameters estimation is in Sec ~\ref{sec:fit}. The results and conclusions are discussed in Sec.~\ref{sec:result} and Sec.~\ref{sec:conclusion}.  

